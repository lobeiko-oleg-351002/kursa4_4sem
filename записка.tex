\documentclass[14pt,a4paper]{extreport}
\usepackage[top=2cm, left=3cm, bottom=2cm, right=1cm]{geometry}
\usepackage[utf8x]{inputenc} % Включаем поддержку UTF8
\usepackage[russian]{babel} % Пакет поддержки русского языка
\usepackage{lscape}
\usepackage{fancyhdr}
\usepackage{textcase}


\title{}
\author{}

\begin{document}
%----------ТИТУЛЬНЫЙ-ЛИСТ--------------
	\center
	Министерство образования Республики Беларусь\\
	Учреждение образования «Белорусский государственный университет информатики и радиоэлектроники»
	\vspace*{2cm}
	\endcenter
	\raggedright
	Факультет компьютерных систем и сетей\\
	\medskip
	Кафедра программного обеспечения информационных технологий\\
	\medskip
	Дисциплина:  Компьютерные системы и сети (КСиС)
	\vspace*{2cm}
	\center
	ПОЯСНИТЕЛЬНАЯ ЗАПИСКА\\
	к курсовому проекту\\
	на тему\\
	\medskip
	СЕТЕВОЕ ИГРОВОЕ ПРИЛОЖЕНИЕ \\
	\medskip
	БГУИР КП  1-40 01 0 15 ПЗ
	\vspace*{4cm}
	\endcenter
	\raggedright
	\hspace*{7.94cm}Студент:  гр. 351002 Лобейко О. М.\\
	\bigskip
	\hspace*{7.94cm}Руководитель: асс. Третьяков Ф.И.\\
	\center
	\vspace*{2cm}
	Минск 2015
	\pagestyle{empty}
%-------ЛИСТ-ЗАДАНИЯ--------------
	\newpage
	\center
	Учреждение образования\\
	\medskip
	«Белорусский государственный университет информатики и радиоэлектроники»\\
	\medskip
	Факультет компьютерных систем и сетей\\
	\medskip
	\endcenter
	\raggedright
	\hspace*{9.53cm}УТВЕРЖДАЮ\\
	\hspace*{9.53cm}Заведующий кафедрой ПОИТ\\
	\hspace*{9.53cm}\underline{\hspace{6cm}} \\
	\hspace*{11cm}\small (подпись) \normalsize\\
	\hspace*{9.53cm}\underline{\hspace{5cm}}2015 г.\\
	\medskip
	\center
	ЗАДАНИЕ\\
	по курсовому проектированию\\
	\medskip
	\endcenter
	\raggedright
	Студенту \underline{Лобейко Олегу Михайловичу}\\
	\begin{enumerate}
	\item Тема работы \underline{Сетевое игровое приложение}\\ 
	\item Срок сдачи студентом законченной работы \underline{DD.MM.YYYY}
	\item Исходные данные к работе \underline{C-Sharp}
	\item Содержание расчётно-пояснительной записки (перечень вопросов, которые подлежат разработке)\\
	\underline{\hspace*{16cm}}\hspace*{-16cm}Введение. 1. Постановка задач 2. Аналитический обзор. \\
	\underline{\hspace*{16cm}}\hspace*{-16cm}3.Разработка приложения. 4. Руководство пользователя.\\
	\underline{\hspace*{16cm}}\hspace*{-16cm}Заключение. Приложения.
	\item Перечень графического материала (с точным обозначением обязательных чертежей и графиков)\\
	\underline{1. Схема алгоритма}
	\item Консультант по курсовой работе\\
	\underline{Третьяков Ф.И.}  
	\item Дата выдачи задания \underline{DD.MM.YYYY}
	\item Календарный график работы над проектом на весь период проектирования (с обозначением сроков выполнения и процентом от общего объёма работы):\\
	\underline{\hspace*{16cm}}\hspace*{-16cm}раздел 1 к DD.MM.YYYY – 15 \% готовности работы;\\  
	\underline{\hspace*{16cm}}\hspace*{-16cm}разделы 2, 3 к DD.MM.YYYY – 30 \% готовности работы;\\ 
	\underline{\hspace*{16cm}}\hspace*{-16cm}раздел 4 к DD.MM.YYYY – 60 \% готовности работы;\\
	\underline{\hspace*{16cm}}\hspace*{-16cm}раздел 5, 6 к DD.MM.YYYY  –  90 \% готовности работы;\\
	\underline{\hspace*{16cm}}\hspace*{-16cm}оформление пояснительной записки и графического материала к\\
	\underline{\hspace*{16cm}}\hspace*{-16cm}DD.MM.YYYY – 100 \% готовности работы.\\
	\underline{\hspace*{16cm}}\hspace*{-16cm}Защита курсового проекта с DD по DD декабря YYYY г.\\
	\end{enumerate}
	\hspace*{7cm}РУКОВОДИТЕЛЬ\underline{\hspace*{6cm}}\hspace*{-3.9cm}Третьяков Ф.И.\\
	\hspace*{11.5cm}\small (подпись) \normalsize\\
	\bigskip
	Задание принял к исполнению \underline{\hspace*{10.5cm}}\hspace*{-8cm} Лобейко О.М. DD.MM.YYYYг.\\
	\hspace*{7cm}\small (дата и подпись студента) \normalsize\\
	%-------СОДЕРЖАНИЕ--------------
	\newpage
	\pagestyle{plain}
	%\renewcommand{\headrulewidth}{0px}
	%\fancypagestyle{plain}{\cfoot{}\rfoot{\thepage}}
	
	\renewcommand\contentsname{\center\normalsize \textbf{СОДЕРЖАНИЕ} \endcenter}
	\tableofcontents
	\endcenter
	%-----ВВЕДЕНИЕ-----
	\newpage
	\addcontentsline{toc}{section}{ВВЕДЕНИЕ}
	\section*{\center\normalsize ВВЕДЕНИЕ \endcenter}
	\hspace{4ex}Фáйтинг (от англ. Fighting — бой, драка, поединок, борьба) — жанр компьютерных игр, имитирующих рукопашный бой малого числа персонажей в пределах ограниченного пространства, называемого ареной.

\hspace{4ex}Файтинги близки к играм жанра «Избей их всех», однако между ними существуют различия. Так, в большинстве файтингов игроку не требуется перемещаться по длинному уровню и нельзя выйти за границы арены, а бой состоит из нечётного числа отдельных раундов и не является непрерывным. Менее значительными и необязательно присутствующими признаками жанра являются использование многочисленных шкал для изображения жизненно важных показателей персонажей и прорисовка бойцов на арене в профиль.\\

\hspace{4ex}Важной особенностью файтингов является их нацеленность на соревнование, а не на сотрудничество игроков, что делает игры этого жанра подходящими для киберспортивных чемпионатов. Обычно файтинги предоставляют игроку возможность вести бой в режиме «один на один» против компьютерного противника или другого игрока, реже — позволяют сражаться одновременно трём или четырём противникам на одной арене.\\

\hspace{4ex}Файтинги, являясь аркадным жанром, получали распространение и пользовались популярностью преимущественно в тех странах, где присутствовала развитая сеть игровых залов с аркадными автоматами, в первую очередь США и Японии. По мере расширения рынка игровых приставок файтинги портировались и на них. На персональных компьютерах игры этого жанра представлены слабо.\\

	\hspace{4ex}В этом приложении будет совмещен жанр файтинга с другим  жанром, сыскавшему популярность в 90-х - платформером. Сражение будет происходить на довольно протяженном для файтингом уровне между большим количеством игроков. Основной идеей является выбор в качестве бойцов известных персонажей из разных игр.

	%-----ПОСТАНОВКА ЗАДАЧ----
	\newpage
	\addcontentsline{toc}{section}{1 ПОСТАНОВКА ЗАДАЧ}
	\section*{\normalsize\hspace{4ex}1 ПОСТАНОВКА ЗАДАЧ}
	\hspace{4ex}На текущем этапе разработки стоят следующие задачи:
	\\1. Реализовать управление игровым персонажем, перемещение персонажа по игровому уровню посредством команд, вводимых с клавиатуры. 
	\\2. Реализовать анимацию персонажа с помощью спрайтов. 
	\\3. Построить игровой уровень.
	\\4. Разработать алгоритмы взаимодействия игрового персонажа с платформами. 
	\\5. Реализовать возможность подключения игроков по локальной сети.
	\\6. Разработать алгоритмы взаимодействия персонажей между собой.

	%-----АНАЛИТИЧЕСКИЙ ОБЗОР----
	\newpage
	\addcontentsline{toc}{section}{2 АНАЛИТИЧЕСКИЙ ОБЗОР}
	\section*{\normalsize\hspace{4ex}2 АНАЛИТИЧЕСКИЙ ОБЗОР}
	\addcontentsline{toc}{subsection}{2.1 Анализ литературных источников}
	\subsection*{\normalsize\hspace{4ex}2.1 Анализ литературных источников}
	\hspace{4ex}Simple DirectMedia Layer (SDL) — это свободная кроссплатформенная мультимедийная библиотека, реализующая единый программный интерфейс к графической подсистеме, звуковым устройствам и средствам ввода для широкого спектра платформ. Данная библиотека активно используется при написании кроссплатформенных мультимедийных программ (в основном игр). Основная часть SDL содержит базовый, весьма ограниченный спектр возможностей. Дополнительную функциональность обеспечивают библиотеки расширений, которые обычно входят в поставку SDL[7].
\\\hspace{4ex}Спрайт — графический объект в компьютерной графике. Чаще все-го — растровое изображение, свободно перемещающееся по экрану. Изна-чально под спрайтами понимали небольшие рисунки, которые выводились на экран с применением аппаратного ускорения. На некоторых машинах (MSX 1, NES) программная прорисовка приводила к определённым ограничениям, а аппаратные спрайты этого ограничения не имели. Впоследствии с увеличением мощности центрального процессора, от аппаратных спрайтов отказались, и понятие «спрайт» распространилось на всех двумерных персонажей[8]. 
Рендеринг — термин в компьютерной графике, обозначающий процесс получения изображения по модели с помощью компьютерной программы[9].
\\\hspace{4ex}Спрайт-лист – графический файл, в котором хранятся спрайты для статический игровых объектов и наборы спрайтов для анимации подвижных игровых объектов.
\\\hspace{4ex}Технология коллизии (столкновения объектов)
\\В двухмерных платформерах 80-х и начала 90-х  (таких как Super Mario) используется тайловая (англ. Tile – плитка) система, т.е. положение платформ, а значит и вся карта уровня, задаётся в двухмерном массиве. Таким образом, уровень представляет собой поле, разбитое на клетки, которые характеризуются наличием в них платформ. Такая технология не требует много ресурсов для вычислений, однако ограничивает разнообразие уровней, превращая игру в набор неподвижных прямоугольников, где единственные динамические объекты – это игрок и враги.
\\\hspace{4ex}Существует технология пиксельной коллизии, где каждому объекту соответствует его ColliderBox (бокс) , в котором объект представляется в виде полосок разной длины и ширины. Такая технология использовалась в играх серии Sonic, Castlevania, Contra HardCorps.  Бокс хранит координаты первых пикселей в строке, которые представлены в программе с помощью векторов. Таким образом, бокс представляет собой массив векторов.  С помощью бокса можно представить объекты с любым контуром. В каждом такте главного цикла определённый метод проверяет на совпадение координат элемента вектора персонажа с элементами векторов других объектов. Пиксельная коллизия работает медленнее, но даёт возможность создавать наклонные и движущиеся объекты как экземпляры соответствующих классов.
\\\hspace{4ex}Проанализировав две технологии, реализуем комбинированную систему: применим тайловую систему для статических платформ и пиксельную коллизию для персонажа, движущихся и наклонных поверхностей. Таким образом, уровни станут разнообразнее и сложнее.

	\addcontentsline{toc}{subsection}{2.2 Обоснование выбора информационных технологий}
	\subsection*{\normalsize\hspace{4ex}2.2 Обоснование выбора информационных технологий}
	\hspace{4ex}1)	Выбор языка программирования
\\Для реализации поставленных задач требуется использование элементов объектно-ориентированного программирования (ООП) и работа с сетевыми технологиями. Язык С-Sharp обладает необходимым набором инструментов для выполнения поставленных задач
\\\hspace{4ex}2)	Использование сторонних библиотек
\\Разработка игровых приложения посредством Windows API (англ. application programming interfaces) является неоправданно трудоёмкой ра-ботой в связи с обращением напрямую к операционной системе Windows. Для обеспечения всех необходимых аппаратных и программных функций достаточно использования низкоуровневой библиотеки Simple DirectMedia Layer (SDL), что позволяет не углубляться в изучение WinAPI , при этом сохраняя низкоуровневую точность разработки.
\\\hspace{4ex} 3)	Сетевые технологии
\\\hspace{4ex}Для реализации сетевой игры будет использоваться TCP/IP соединение с помощью сокетов, что позволяет устанавливать соединение с неограниченным числом пользователей в локальной сети.

\addcontentsline{toc}{subsection}{2.3 Сравнение с существующими аналогами}
	\subsection*{\normalsize\hspace{4ex}2.3 Сравнение с существующими аналогами}
\hspace{4ex} Если сравнивать данный проект с такими файтингами, как Mortal Kombat, Street Fighter, Tekken, то можно сказать, что в этом проекте нету сложной системы повреждений (коллайдер на персонажа всего один). Игровой процесс больше похож на платформер, где можно перемещаться по уровню, чего в аналогах делать нельзя, так как арены там представляют собой плоскость. В этой игре можно строить уровни из блоков, что вводит некоторую тактическую составляющую в игру. Так же есть возможность подключения большого количества игроков в то время, как в аналогах число игроков ограничевается двумя.

	%-----РАЗРАБОТКА ПРИЛОЖЕНИЯ------
	\newpage
	\addcontentsline{toc}{section}{3 РАЗРАБОТКА ПРИЛОЖЕНИЯ}
	\section*{\normalsize\hspace{4ex}3 РАЗРАБОТКА ПРИЛОЖЕНИЯ}
	\addcontentsline{toc}{subsection}{3.1 Структура данных}
	\subsection*{\normalsize\hspace{4ex}3.1 Структура данных}
	\hspace{4ex}
\\1)	Входные данные
\\Входные данные представляют собой нажатие клавиш на клавиатуре и мыши. 
\\2)	Выходные данные
\\Информация о герое, местоположении героя на уровне, структуре уровня и производимых действиях отображается графически в окне игрового приложения, необходимые другим клиентам о персонаже данные передаются хосту.
	\addcontentsline{toc}{subsection}{3.2 Анализ требований и разработка необходимых модулей}
	\subsection*{\normalsize\hspace{4ex}3.2 Анализ требований и разработка необходимых модулей}
	\hspace{4ex}В программе должны быть разработаны следующие классы:
\\1)	Класс текстур  – класс, который отвечает за загрузку из файла, хранение и рендеринг текстур объекта (персонажа, платформы).
\\Используемые методы:
\\loadFromFile – загружает текстуру из спрайт-листа;
\\ render – отрисовывает текстуру объекта на экране по заданным координатам из вызываемого объекта;
\\free – освобождает память, занятую текстурой[10].
\\2)	Класс статической платформы – класс, который отвечает за отображение статической платформы на экране.
\\Используемые методы:
\\loading – вызывает метод loadFromFile класса Texture, передавая постоянные координаты конкретного объекта в спрайт-листе;
\\render – вызывает метод render класса Texture, передавая коорди-наты текущего объекта на уровне и размеры спрайта;
\\close – вызывает метод free класса Texture для текущей текстуры.
\\3)	Класс персонажа (Prototype) – класс, который отвечает за управле-ние персонажем, его рендеринг и взаимодействие с игровым уровнем.
\\Используемые методы:
\\loading - вызывает метод loadFromFile класса Texture, передавая по-стоянные координаты спрайтов для создания анимаций объекта в спрайт-листе;
\\ColisWithWalls – проверяет, столкнулся ли персонаж со статической платформой;
\\handleEvent – осуществляет управление персонажем с помощью ввода с клавиатуры;
\\move – осуществляет перемещение персонажа по игровому уровню;
\\render -  вызывает метод render класса Texture, передавая координаты персонажа на уровне и размеры спрайта;
\\ColisWithPlayers - проверяет на столкновение персонажей;
\\методы на получение текущей анимации (абстрактные);
\\методы на получение текущего коллайдера (абстрактные);
\\close – вызывает метод free класса Texture для текущей текстуры.
\\4)	Класс героя - наследуемый класс от персонажа, представляющий собой частную реализацию его абстрактных методов;


	%-----РУКОВОДСТВО ПОЛЬЗОВАТЕЛЯ----
	\newpage
	\addcontentsline{toc}{section}{4 РУКОВОДСТВО ПОЛЬЗОВАТЕЛЯ}
	\section*{\normalsize\hspace{4ex}4 РУКОВОДСТВО ПОЛЬЗОВАТЕЛЯ}
	\hspace{4ex}инструкция по игре и сетевому подключению
	%-------ЗАКЛЮЧЕНИЕ-------
	\newpage
	\addcontentsline{toc}{section}{ЗАКЛЮЧЕНИЕ}
	\section*{\center\normalsize ЗАКЛЮЧЕНИЕ \endcenter}
	\hspace{4ex}что сделано и что планируется
	%----СПИСОК ИСПОЛЬЗОВАННОЙ ЛИТЕРАТУРЫ-------
	\newpage
	\addcontentsline{toc}{section}{СПИСОК ИСПОЛЬЗОВАННОЙ ЛИТЕРАТУРЫ}
	\section*{\center\normalsize СПИСОК ИСПОЛЬЗОВАННОЙ ЛИТЕРАТУРЫ \endcenter}
	\hspace{4ex}
\\		Вирт, Н. Алгоритмы и структуры данных / Н.Вирт. – М. : Мир, 1989.
\\	Кнут, Д. Искусство программирования для ЭВМ. В 3 т.  Т.2 / Д. Кнут. – М. : Мир, 1978. 
\\	А. В. Ахо, Р. Сети, Д. Д. Ульман. Компиляторы: принципы, технологии и инструменты. М.: «Вильямс», 2003. С. 51.  
\\	ГОСТ 19.701–90. Единая система программной документации. Схемы алгоритмов, программ, данных и систем. Условные обозначения и правила выполнения. – Введ. 1992–01–01. – М. : Изд-во стандартов, 1991. 
\\	Роберт Мартин «Чистый код. Создание, анализ и рефакторинг» Питер, 2014.


	%----ПРИЛОЖЕНИЕ А (обязательное) Исходный код программы----
	\begin{landscape}
	\newpage
	\addcontentsline{toc}{section}{ПРИЛОЖЕНИЕ А}
	\section*{\center\normalsize ПРИЛОЖЕНИЕ А\\(обязательное)\\Исходный код программы \endcenter}
	код
	\end{landscape}
	
	
\end{document}          
